% Общие поля титульного листа диссертации и автореферата
\institution{Название организации}

\topic{Тема диссертации}

\author{ФИО автора}

\specnum{01.04.05}
\spec{Оптика}
%\specsndnum{01.04.07}
%\specsnd{Физика конденсированного состояния}

\sa{ФИО руководителя}
\sastatus{д.~ф.-м.~н., проф.}
%\sasnd{ФИО второго руководителя}
%\sasndstatus{к.~ф.-м.~н., проф.}

%\scon{ФИО консультанта}
%\sconstatus{д.~ф.-м.~н., проф.}
%\sconsnd{ФИО второго консультанта}
%\sconsndstatus{д.~ф.-м.~н., проф.}

\city{Санкт-Петербург}
\date{\number\year}

% Общие разделы автореферата и диссертации
\mkcommonsect{actuality}{Актуальность темы исследования.}{%
Текст об актуальности. Ссылка~\cite{Yoffe_1993_AP_42_173}.
}

\mkcommonsect{development}{Степень разработанности темы исследования.}{
Текст о степени разработанности темы.
}

\mkcommonsect{objective}{Цели и задачи диссертационной работы:}{%
Список целей.

Для достижения поставленных целей были решены следующие задачи:
}

\mkcommonsect{novelty}{Научная новизна.}{%
Текст о новизне.
}

\mkcommonsect{value}{Теоретическая и практическая значимость.}{%
Результаты, изложенные в диссертации, могут быть использованы для ...
}

\mkcommonsect{methods}{Методология и методы исследования.}{%
Текст о методах исследования.
}

\mkcommonsect{results}{Положения, выносимые на защиту:}{%
Текст о положениях и результатах.
}

\mkcommonsect{approbation}{Степень достоверности и апробация результатов.}{%
Основные результаты диссертации докладывались на следующих конференциях:
}

\mkcommonsect{pub}{Публикации.}{%
Материалы диссертации опубликованы в $N$ печатных работах, из них $n_1$
статей в рецензируемых журналах~\cite{Ivanov_1999_Journal_17_173,
Petrov_2001_Journal_23_12321,Sidorov_2002_Journal_32_1531}, $n_2$ статей в
сборниках трудов конференций и $n_3$ тезисов докладов.
}

\mkcommonsect{contrib}{Личный вклад автора.}{%
Содержание диссертации и основные положения, выносимые на защиту, отражают персональный вклад автора в опубликованные работы.
Подготовка к публикации полученных результатов проводилась совместно с соавторами, причем вклад диссертанта был определяющим. Все представленные в диссертации результаты получены лично автором.
}

\mkcommonsect{struct}{Структура и объем диссертации.}{%
Диссертация состоит из введения, обзора литературы, $n$ глав, заключения и библиографии.
Общий объем диссертации $P$ страниц, из них $p_1$ страницы текста, включая $f$ рисунков.
Библиография включает $B$ наименований на $p_2$ страницах.
}
