\documentclass[%
doctor,      % тип документа
subf,        % использовать пакет subcaption для вложенной нумерации рисунков
href,        % использовать пакет hyperref для создания гиперссылок
colorlinks,  % цветные гиперссылки
%times,      % шрифт Times как основной
%fixint,     % включить прямые знаки интегралов
%classified, % гриф секретности
%facsimile,  % отображать факсимиле диссертанта
]{disser}

\usepackage[
  a4paper, mag=1000,
  left=2.5cm, right=1cm, top=2cm, bottom=2cm, headsep=0.7cm, footskip=1cm
]{geometry}

\usepackage[intlimits]{amsmath}
\usepackage{amssymb,amsfonts}
\usepackage[autostyle]{csquotes}

\usepackage[T2A]{fontenc}
\usepackage[utf8]{inputenc}
\usepackage[english,russian]{babel}
\ifpdf\usepackage{epstopdf}\fi

% Шрифт Times в тексте как основной
%\usepackage{tempora}
% альтернативный пакет (из дистрибутива TeX Live)
%\usepackage{cyrtimes}

% Шрифт Times в формулах как основной
%\usepackage[varg,cmbraces,cmintegrals]{newtxmath}
% альтернативный пакет
%\usepackage[subscriptcorrection,nofontinfo]{mtpro2}

\usepackage[style=gost-numeric,
  backend=biber,
  language=auto,
  hyperref=auto,
  autolang=other,
  sorting=none
]{biblatex}

\addbibresource{thesis.bib}

% Номера страниц снизу и по центру
%\pagestyle{footcenter}
%\chapterpagestyle{footcenter}

% Точка с запятой в качестве разделителя между номерами цитирований
%\setcitestyle{semicolon}

% Ссылки на работы соискателя включаются в общий список литературы
\let\citemy=\cite

% Использовать полужирное начертание для векторов
\let\vec=\mathbf

% Путь к файлам с иллюстрациями
\graphicspath{{fig/}}

\begin{document}

% Переопределение стандартных заголовков
%\def\contentsname{Содержание}
%\def\conclusionname{Выводы}
%\def\bibname{Литература}

% Включение файла с общим текстом диссертации и автореферата
% (текст титульного листа и характеристика работы).
% ����� ���� ���������� ����� ����������� � ������������
\institution{�������� �����������}

\topic{���� �����������}

\author{��� ������}

\specnum{01.04.05}
\spec{������}
%\specsndnum{01.04.07}
%\specsnd{������ ����������������� ���������}

\sa{��� ������������}
\sastatus{�.~�.-�.~�., ����.}
%\sasnd{��� ������� ������������}
%\sasndstatus{�.~�.-�.~�., ����.}

%\scon{��� ������������}
%\sconstatus{�.~�.-�.~�., ����.}

\city{�����-���������}
\date{\number\year}

% ����� ������� ������������ � �����������
\mkcommonsect{actuality}{������������ ������}{%
����� �� ������������. ������~\cite{Yoffe_1993_AP_42_173}.
}

\mkcommonsect{objective}{���� ��������������� ������}{%
������� � ...

��� ���������� ������������ ����� ���� ������ ��������� ������:

}

\mkcommonsect{novelty}{������� �������}{%
����� � �������.
}

\mkcommonsect{value}{������������ ����������}{%
����������, ���������� � �����������, ����� ���� ������������ ��� ...
}

\mkcommonsect{results}{%
�� ������ ��������� ��������� �������� ���������� � ���������:}{%
����� � �����������.
}

\mkcommonsect{approbation}{��������� ������}{%
�������� ���������� ����������� ������������� �� ��������� ������������:
}

\mkcommonsect{pub}{����������.}{%
��������� ����������� ������������ � $N$ �������� �������, �� ��� $n_1$
������ � ������������� ��������~\citemy{Ivanov_1999_Journal_17_173,
Petrov_2001_Journal_23_12321,Sidorov_2002_Journal_32_1531}, $n_2$ ������ �
��������� ������ ����������� � $n_3$ ������� ��������.
}

\mkcommonsect{contrib}{������ ����� ������}{%
���������� ����������� � �������� ���������, ��������� �� ������, �������� ������������ ����� ������ � �������������� ������.
���������� � ���������� ���������� ����������� ����������� ��������� � ����������, ������ ����� ����������� ��� ������������. ��� �������������� � ����������� ���������� �������� ����� �������.
}

\mkcommonsect{struct}{��������� � ����� �����������}{%
����������� ������� �� ��������, ������ ����������, $n$ ����, ���������� � ������������.
����� ����� ����������� $P$ �������, �� ��� $p_1$ �������� ������, ������� $f$ ��������.
������������ �������� $B$ ������������ �� $p_2$ ���������.
}


% номер копии для грифа секретности
%\copynum{1}
% класс доступа
%\classlabel{Для служебного пользования}

% номер УДК
\libcatnum{12345}

\title{ДИССЕРТАЦИЯ\\
на соискание ученой степени\\
доктора физико-математических наук}

\maketitle

%%
%% Titlepage in English
%%
%
%\institution{Name of Organization}
%
%\title{Doctoral Dissertation}
%
%% Topic
%\topic{Dummy Title}
%
%% Author
%\author{Author's Name}
%
%\specnum{01.04.05}
%\spec{Optics}
%
%%\specsndnum{01.04.07}
%%\specsnd{Condensed matter physics}
%
%% Scientific consultants
%\scon{B.\,B.~Baranov}
%\sconstatus{Professor}
%%\sconsnd{P.\,P.~Petrov}
%%\sconsndstatus{Professor}
%
%% City & Year
%\city{Saint Petersburg}
%\date{\number\year}
%
%\maketitle[en]

% Содержание
\tableofcontents

% Введение
\intro

%
% ������������ ����� ������� ������������ � ����� common.tex.
%

% ����� ���� ���������� ����� ����������� � ������������
\institution{�������� �����������}

\topic{���� �����������}

\author{��� ������}

\specnum{01.04.05}
\spec{������}
%\specsndnum{01.04.07}
%\specsnd{������ ����������������� ���������}

\sa{��� ������������}
\sastatus{�.~�.-�.~�., ����.}
%\sasnd{��� ������� ������������}
%\sasndstatus{�.~�.-�.~�., ����.}

%\scon{��� ������������}
%\sconstatus{�.~�.-�.~�., ����.}

\city{�����-���������}
\date{\number\year}

% ����� ������� ������������ � �����������
\mkcommonsect{actuality}{������������ ������}{%
����� �� ������������. ������~\cite{Yoffe_1993_AP_42_173}.
}

\mkcommonsect{objective}{���� ��������������� ������}{%
������� � ...

��� ���������� ������������ ����� ���� ������ ��������� ������:

}

\mkcommonsect{novelty}{������� �������}{%
����� � �������.
}

\mkcommonsect{value}{������������ ����������}{%
����������, ���������� � �����������, ����� ���� ������������ ��� ...
}

\mkcommonsect{results}{%
�� ������ ��������� ��������� �������� ���������� � ���������:}{%
����� � �����������.
}

\mkcommonsect{approbation}{��������� ������}{%
�������� ���������� ����������� ������������� �� ��������� ������������:
}

\mkcommonsect{pub}{����������.}{%
��������� ����������� ������������ � $N$ �������� �������, �� ��� $n_1$
������ � ������������� ��������~\citemy{Ivanov_1999_Journal_17_173,
Petrov_2001_Journal_23_12321,Sidorov_2002_Journal_32_1531}, $n_2$ ������ �
��������� ������ ����������� � $n_3$ ������� ��������.
}

\mkcommonsect{contrib}{������ ����� ������}{%
���������� ����������� � �������� ���������, ��������� �� ������, �������� ������������ ����� ������ � �������������� ������.
���������� � ���������� ���������� ����������� ����������� ��������� � ����������, ������ ����� ����������� ��� ������������. ��� �������������� � ����������� ���������� �������� ����� �������.
}

\mkcommonsect{struct}{��������� � ����� �����������}{%
����������� ������� �� ��������, ������ ����������, $n$ ����, ���������� � ������������.
����� ����� ����������� $P$ �������, �� ��� $p_1$ �������� ������, ������� $f$ ��������.
������������ �������� $B$ ������������ �� $p_2$ ���������.
}


% ������������ ������
\actualitysection
\actualitytext

% ���� ��������������� ������
\objectivesection
\objectivetext

% ������� �������
\noveltysection
\noveltytext

% ������������ ��������
\valuesection
\valuetext

% ���������� � ���������, ��������� �� ������
\resultssection
\resultstext

% ��������� ������
\approbationsection
\approbationtext

% ����������
\pubsection
\pubtext

% ������ ����� ������
\contribsection
\contribtext

% ��������� � ����� �����������
\structsection
\structtext

% Обзор литературы
%\review


% Основная часть
%% Глава 1
\chapter{Название главы}
\section{Название секции}
\section{Выводы к первой главе}
%% Глава 2
%\input{2}

% Заключение
\conclusion


% Список сокращений и условных обозначений
%% ������ ���������� � �������� �����������
\defs

\begin{longtable}{lp{0.025\textwidth}p{0.89\textwidth}}
$\vec A$ & --- & ��������� ���������\\
$I$ & --- & ������������� �����\\
$j_l$ & --- & ����������� ������� ������� ������� ����\\
$Y_{lm}$ & --- & ����������� ���������\\
\end{longtable}


% Словарь терминов
%% ������� ��������
\dict

\textbf{������} "--- �����������.


% Список литературы
\printbibliography[heading=bibintoc]

% Список иллюстративного материала
%\listoffigures

% Приложения
%\appendix
%\appendix
\chapter{�������� ����������}


\end{document}
